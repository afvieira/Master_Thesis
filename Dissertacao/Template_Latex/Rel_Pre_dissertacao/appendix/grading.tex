\section{Automatic Grading}\label{s:grading}

For a better comprehension of this summary and the grading process, let us now introduce one example of a contest where \textsf{Competitors} are invited to solve several mathematical problems. %, like the one proposed on Figure~\ref{fig:enunc}.
The challenge is to present a solution to calculate the \emph{n$^{th}$} fibonacci number, where \emph{n} is a number asked to the user.
In Figure~\ref{fig:answers}, we are comparing two different solutions proposed by two \textsf{Teams}: the \textsf{Rationals} on the left-hand size, and \textsf{Smarties} on the right-hand size.

The first solution, assessed by the system with "Solved Some Tests" (as we can see in Figure~\ref{fig:problem}), only calculates correctly the first and second numbers of the sequence (0 and 1) and, consequently only passes in 22\% of the tests (2 in 9).
This leads to a final score on the \emph{Execution} field of only 22.2\%. 
In a closer look, we can conclude that this answer is well documented (it has 15 commented lines in 50 lines of code) and its \emph{Legibility} has a final score of 93\%.
Its \emph{Complexity} score of 73\% is owed to its number of used variables and data structures.
Its 50 lines of code exceed the average size of the contest submissions for this problem, which penalizes its final grade on the \emph{Dimension} category (only 45.2\%).
In the \emph{Consistency} category, its 21.6\% are owed to the use of several returns through the code (assuming a maximum of two returns per function as a reasonable limit to the source code consistency) and also to the use of pointers.

The second one, assessed by the system with "Solved" (as depicted in Figure~\ref{fig:problem}), follows an iterative strategy to implement the Fibonacci recurrence. 
This answer is better documented (11 commented lines in 36 lines of code), but its \emph{Legibility} has a lower final score (88\%) once it did not used \emph{defines}, as the other solution.
Its \emph{Complexity} score of 58\% is owed to the implemented \emph{loop}.
Its 36 lines of code improve its final grade on the \emph{Dimension} category to 88.6\%.
In the \emph{Consistency} category, its weak 22\% are also owed to the use of several returns through the code.

Finally we can conclude that both solutions have not been plagiarized (100\% original) and \emph{Competitors} follow the good practice of no repeating their own code (0\% of duplicated code).
These assessments led to a final grade of 29\% on the first case and a significant 89\% on the second case.
\quim\ also completes this evaluation with radial charts to a quicker and easier comparison of the solution performance in the different grading categories.
