
\section{Actors and Roles}\label{ss:profiles}

\quim\ provides different user interfaces targeted to the  profiles
that reflect the roles of the different contest participants, as descried bellow.

\begin{itemize}
\item \textsf{Administrator} -- this view has two modes:  \emph{Administration} and  \emph{General}.
The first one is used to access the back-end area to setup the system data and make it operational.
%It allows to manage all the programming contest tasks, including the contest creation 
%and management of the contest itself.
The second one allows to participate in a contest as a general user.
%according to its role on it (Teacher, Competitor or Judge).

\item \textsf{Teacher} -- this view allows a user to publish problems in a contest, add a set of tests.
%follow the contest flow and consult contest statistics.

\item \textsf{Competitor} -- used for the player associate himself to a \textsf{Contest} and a \textsf{Team} and compete.
%To play, the user uses this interface to read the problem statements and to submit the answers.
%The view is also used to receive feedback about the compilation and the acceptance of his submission.
%(according to the set of tests passed successfully).


\item \textsf{Judge} -- used to act as a decision maker, giving feedback about the team submissions. 
%after a manual verification of their automatic grade.

\item \textsf{Guest} -- this mode allows any user (not involved in any of the previous roles)
to follow the progress of a contest in an anonymous mode (without being registered).

\end {itemize}

For security purposes and to ensure that only authorized users access the system,
all the user profiles are accessible by authentication (except for the \textsf{Guest} mode).
For this, it is mandatory that users register themselves on the system.
The Administrator may posteriorly define new permissions for \textsf{Teachers} and \textsf{Judges} in each contest.
